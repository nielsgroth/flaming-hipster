\chapter{Conclusion} 
\label{chap:conclusion}
This thesis has studied the behaviour of Cyclon in realistic conditions on a
real network. The different metrics of complex networks have been discussed and
the measured results have helped to better understand, how Cyclon behaves under
various conditions. The use of a wireless ad-hoc network has helped to examine
the impact of package loss and latency on Cyclon.  Some of the experiments
conducted redeemed to be of little use, due to the small network size. Still,
the remaining experiments have shown, that Cyclon copes well under harsh network
conditions. With moderate package loss Cyclon shows to perform slightly worse
than in theory, but the impact is not breaking the basic functionality.
Latencies can pose a problem in rare cases, but the protocol could be altered,
so that those cases add to the package loss. A high churn rate is met with
Cyclon's capability to quickly react to network changes. During the
implementation of Cyclon in JAVA it showed to have a minor problem with
latency, that cannot be solved to full satisfaction. The same was discovered
in \cite{matoslaystream}. Despite this weaknesses, its overall robustness in
face of moderate package loss and a high churn rate showed to be very good. Paired
with the simplicity and light weight, Cyclon shows good properties and it has
shown that it withstands the moderate package loss and high churn rates of the
experiment environment. Another important contribution is the statistics
gathering framework, which can be used for other topology related experiments,
with any programming language that supports web services.
From the findings of this thesis Cyclon has to be be improved, by changing the
way it behaves in in the case of incoming shuffle requests while the node is
waiting for a response. This could not have been discovered without considering
network latencies.

To my knowledge, there has been no experiments yet, where peer sampling
services have been tested under realistic network conditions to this extend.
Most of the experiments have been simulated with dedicated peer-to-peer
simulators like \cite{montresor2009peersim}. In this light a major contribution
of this thesis is to show the importance of package loss and network latency,
when considering gossiping algorithms. The science field of gossiping protocols
would greatly benefit from an experiment framework, that takes into
consideration the conditions, that a physical network produces. As a
contribution the statistics server introduced and implemented in this thesis, takes a step
towards such a framework.
