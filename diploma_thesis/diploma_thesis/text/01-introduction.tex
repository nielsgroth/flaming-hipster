\chapter{Introduction} 
\label{chap:introduction}

\section{Introduction}
% What is the problem? Why is it important? 
Since their introduction in the late 1990's peer-to-peer Networks have become an
important part of the Internet. Multiple peer-to-peer based applications like
file sharing, real time audio or video streaming and others make up for an average of
79\% network traffic and 42\% of connections \cite{John+:InetTraffic}.
Peer-to-Peer networks have very large numbers of participants with high
fluctuations in activity. According to \cite{bittorrent} Bittorrent and
$\mu$Torrent, the two main clients for the Bittorrent file sharing network, had
150 million distinct users in December 2011. Several studies have shown that the network size remains stable,
but that al lot of peer join and leave the network over time. In fact those
studies show that within an hour up to 50\% of the network peers have
changed\cite{stutzbach2006understanding} \cite{falkner2007profiling}. This
dynamic behaviour of the peers is called churn. There are different types of
peer-to-peer systems depending on their level of decentralization. In
unstructured pure peer-to-peer networks it is desired to manage the
infrastructure decentrally. This means that there needs to be a
decentral mechanism to manage the network itself and the service provided by
the network. A crucial part of managing such a network is knowing which peers belong
to it. This is called membership management. Since the networks are very big the
membership managing solution has to be scalable. In order to deal with the high
churn rates present in peer-to-peer networks a solution has to quickly adapt to
changes in the networks to provide a reliable service. For network performance
it is also required that peers can reach other peers in a short time despite
the size of the network. Membership management is mainly the task to keep
information about all participating peers accessible for all peers. This is
done by dividing the information so that every peer stores a partial view of
the network. In membership management this information usually consists out of
the IP address, port and any other protocol specific metrics. The membership
management therefore provides a structure which can also be called an overlay
topology.

Cyclon introduced by \cite{voulgaris2005cyclon} is an inexpensive membership
management algorithm that provides an overlay topology with similar properties
to random graphs \cite{Bollobas1985}. It uses gossiping to introduce randomness
and quick information dissemination. The Cyclon algorithm has so far been only
tested in a simulated environment.

\section{Objective of the Thesis} 
%What is the thesis? WHat is my contribution?
The goal of this thesis is to examine Cyclon's behaviour under realistic
network conditions, including packet loss, latency and high churn rates. The
overlay topology provided by Cyclon is analyzed for its functionality,
reliability and robustness. To achieve this a JAVA implementation of Cyclon
together with a statistics server for topology information gathering is
provided. Experiments are conducted in a wireless ad-hoc network and an
analysis of the results is conducted to examine the impact of packet loss and
latency on the topology. 

 \section{Outline of the Thesis} % alternative title:
% Thesis Structure How is this thesis structures? Which chapter contains what?
This thesis is structured as follows. This chapter \ref{chap:introduction}
introduces the problem and shows what Cyclon can achieve. There is also a quick
survey of the state of the art in unstructured Peer-to-Peer membership
management field. Chapter \ref{chap:gossiping} will describe more deeply where
in peer-to-peer networks Cyclon can be used, what gossiping is and how Cyclon
works basically. In Chapter \ref{chap:properties} the requirements for and
properties of gossiping membership management protocols are listed and
explained. Chapter \ref{chap:methodology} explains how the JAVA implementation
is build. It also describes the network on which the experiments are conducted
and how information about the experiments is gathers by the statistics server.
The experiments results and analysis are provided in chapters \ref{chap:results}
and \ref{chap:evaluation}. Chapter \ref{chap:conclusion} concludes this thesis
and gives and outlook for future development.

\section{State of the Art}
%alternative title:
% related work Lpbcast Newscast

