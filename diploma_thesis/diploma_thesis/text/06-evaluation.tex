\chapter{Analysis of Impact of Network Failures on the Overlay Topology} 
\label{chap:evaluation}

\FloatBarrier
\section{Summary}
	This section contains a summary of the results discussed above.
	The experiments have shown that Cyclon provides an overlay topology that does
	not loose connectivity once established even with moderate package loss and a
	high churn of clients. They have also shown that the topology metrics converge
	to certain values. The typical out-degree in an overlay topology created by
	Cyclon has shown to be close to the cache size. Although with moderate package
	loss the mean out-degree is smaller than without, it is a small difference.
	Most in-degrees in a topology have shown to be close to the cache size if the
	shuffle length was small enough. The experiments node count has shown to be to
	small to show the same for bigger shuffle lengths. The average cluster
	coefficients of all experiments have not been bigger than twice the average
	cluster coefficients of comparable random graphs. Packet loss seems to also
	have a small negative impact on cluster coefficients. The diameter of the
	network is similar to those of small world networks. The betweenness
	centrality for a sufficiently small shuffle length shows that Cyclon does not
	create super nodes, that would be weak points in topology structure. \\
	
	The number of nodes in the experiment has shown to be too small for most of the
	experiment parameters. Especially the shuffle length to cache size ratio has
	proven to have a big effect on in-degree distribution in disfavor of the
	networks symmetry. The networks latency also has shown to have an impact on
	this implementation, in the way that in rare cases the cache size is above its
	initial boundary. Nodes with high in-degree are especially prone to this
	behaviour. The effects of network behaviour will be described in the next
	subsection
	
	
\subsection{Impacts of Package Loss on Topology Development}
	The goal of the experiments was to find what impact package loss has on the
	topology. This is why a wireless network using UDP packets was chosen. The
	shuffle response packet and the shuffle request packet both can get lost. If
	the latter gets lost, this means that in this cycle the node does not make
	a shuffle exchange and that the target node has one less response to send.
	For the whole network this means that there is one less shuffle in the
	network. This slows network development, which has an impact on how fast the
	network converges and repairs itself after big failures.
	If the shuffle response gets lost the impact is larger. The target node has
	exchanged its nodes with the nodes from the shuffle request, but the request
	node cannot. This leads to a change in in-degree distribution in favor of the
	nodes from the shuffle request packet. In extreme cases where nodes are only
	known by one other node and are the only connection to a different part of
	the network, this could lead to a network partitioning. In the experiments
	conducted for this thesis the statistics server counted up to 10\% lost
	response packages. As stated in the diameter section \ref{subsubsec:diameter}
	of this chapter, no partitionings have been observed.\\
	Lost response packets also change the cluster coefficient of the responding
	node. This happens, because after the node exchanges their neighbors to the
	ones from the request packet, it has more neighbors in common with the requesting node,
	which is now also its neighbor. The negative impact on network metric values of
	this scenario also is dependent on the shuffle length to cache size ratio. The
	higher the ratio, the more negative the impact. When the shuffle length is the
	same as the cache size and the caches are full in both nodes and a response
	package gets lost the nodes have almost identical caches. Summarizing a
	moderate packet loss has negative impact on the networks balance, but in the
	experiments did not produce a fatal error.
\subsection{Impacts of Latency on Topology Development}
	Network latencies on the DES-Testbed are very small, compared to the socket
	timeout set in this implementation of Cyclon, but still have an impact on the
	networks behaviour. As described in the section above and the section
	\ref{subsubsec:outDegree} the latency produces an error in cache size
	management. The problem arises, when a node sends a shuffling request and
	before receiving a response, gets another shuffle request. There are several
	possibilities to change Cyclons behaviour in that case. One would be to simply drop
	the package. This leads directly to a higher request package loss ratio.
	Another possibility is to wait for the response before processing the new
	request. While this works fine for the first node it could possibly let the
	timeout of the second requesting node to run out before receiving its response. This
	solution could lead to a higher response package loss. Another option would be,
	for the first node to pick only pointers from the untagged portion of its cache
	to send in the response packet. Depending on the shuffle length to cache size
	ratio this could mean an only partially filled or even empty response packet.
	On receiving an only partially filled package, the second requesting node would
	have to know which nodes it can replace, otherwise pointers could get lost.