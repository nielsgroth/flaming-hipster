\chapter{Measurement Methodology and Implementation} 
\label{chap:methodology}
This chapter describes the architecture of the experiments and the environment
in which it is build. It starts with the physical network on which the
experiments are conducted, the DES-Testbed. The next section is about the Layer
3 routing algorithm used which is the open link state routing deamon for
wireless experiments. The implementation of Cyclon is then described and
explained in the next section and all other environment specifics are given in
the last section of this chapter.

\section{DES-Testbed}
%What is this? How does it work? Why do I use this?
The DES-Testbed is a wireless adhoc mesh network environment developed and
deployed at Freie Universität Berlin to test network protocols for embedded
systems under real conditions. It consists of currently 51 mesh routers each
with 3 wireless network interfaces and 2 ethernet interfaces. The mesh routers
are placed in mostly seperate rooms at the Institute of Computer Science, the
Institute of Mathematics and the Intitute of Physics at the Freie Universtät
Berlin. They are connected through an ethernet network to the main entry point
which in the experiments described in this thesis is also used as a statistics
server for gathering experiment related information. Further the wireless mesh
network is used to introduce network faults like packet loss and a
dynamic layer 3 routing topology to the experiments. The network is also used
for experiments in ethernet environment to obtain comparable results without the
afore mentioned network faults. 
%

\section{OLSR/B.A.T.M.A.N.} %What? How? Why?
%what version of olsrd do i use?
The underlying routing protocol used for the experiments is the OSLR deamon
\cite{Tonnesen2004}.
It is the implementation of the optimized link state routing protocol (OLSR) as
described in \cite{RFC3626}. It creates a decentralized multi-hop routing topology for
wireless mesh networks. It is used successfully in large metropolitan area mesh
networks like the Athens Wireless Metropolitan Network or the Berlin based Freifunk.net.
The basic idea behind it is as follows. To discover its immediate one-hop and
two hop neighbors, a node sends HELLO-messages paired with one-hop neighbor
information. It then chooses a set of multipoint relays (MPR) within its one-hop
neighbors with the best link quality to all of its two-hop neighbors. These are
used for optimized flooding of control packages. Every node also has an MPR
selector list which holds information about all nodes that chose that node
as an MPR. The nodes that are selected as an MPR will now send and forward MPR
selector lists using TC-messages. The OLSR protocol is proactive, which in this
case means it changes its routing topology regularly but not immediatly whenever
a new node appears or an old link to a node is unstable.

\subsection{Txtinfo and Jsoninfo Plugins}
The OLSR deamon created by \cite{Tonnesen2004} also offers the txtInfo plugin
\cite{} to access neighbor, topology and other information from a running OLSR
deamon. It is used by the JsonInfo plugin \cite{} which again creates a Java
interface for access to that information. This is used to obtain information
about the physical topology of the network and for bootstrapping purposes as
described later.

\section{Implementation of Cyclon in Java} %UML-Diagramm

\subsection{The Statistics Server}

The statistics server is used to obtain experiment data and create an outputfile
in the Gephi file format. It provides a web service to obtain the data from the
clients and to create an Gephi readable XML file.  Each node sends routing
topology and the Cyclon neighbor list to the statistics server
through web methods called sendTopology and sendList. The information then is
processed by the statistics server by storing the node along with its neighbor
information in a node object.  It processes the information by storing the
sender as a node updating it's neighbors as edges and The server receives update messages from the clients which it then processes and stores in JAVA objects.
after the experiment is finished it serializes the stored objects to an XML
stream and writes it to a file in the gephi file format described below.

The statistics server application stores all relevant information for an
experiment during its execution. On initialization it creates a collection of
nodes to store the required information about each node as they are born ,
change neighbors and eventually die. The server then publishes its web service
which holds methods to access the server's objects. The main web method is used
by the nodes to update information whenever their status changes. This might be
their birth or the change of their neighbor cache. The node object stores the
information of its first appearance, the last time it got updated and about all
of its edges and at which times they existed. If the node didn't get updated for
specific time it is marked as dead. Should the node ever be updated again it is
reset to alive as if it wasn't dead. All timestamps are stored as nanoseconds in
relation to an arbitrary point in time. The timestamps are given as the
statistics server gets the information so that delays on the network could
distort the results.
The statistics server is implemented as a JAVA web service providing an
interface for storing all information concerning the graph topology during an
experiment. The interface specification is shown below as an UML class diagram
and the web methods included are described below.
%UML class diagram of statistics server class






\section{Experiment Environment}

\subsection{Gephi and Gephi File Format}