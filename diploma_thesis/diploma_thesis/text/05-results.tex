\chapter{Experimental Observations and Result Analysis} 
\label{chap:results}

This chapter provides and discusses all experiment results. In the first part
the conducted experiments are introduced and described. In the second part of
this chapter the resulting overlay topologies are described using several
metrics already introduced in chapter \ref{chap:properties}. The results are
interpreted and where necessary the effects are explained and analyzed.

\section{Experiment Scenarios}
This section describes the different experiment scenarios conducted during the
research for this thesis. The specific experiment parameters are shown in table
\ref{tab:experiments}.
 \begin{table}
\begin{tabular} { |p{2cm}|l|p{1.9cm}|p{2cm}|p{2.5cm}|p{1.9cm}|}
\hline
Experiment Name & Network Mode & Experiment Scenario & Cache Size (neighbors) &
Shuffle length (neighbors) & duration (in cycles)\\
\hline
\hline
B00 & wireless/OLSR & static & 20 & 10 & 1200\\
B01 & wireless/OLSR & static & 20 & 5 & 1200\\
B02 & wireless/OLSR & static & 20 & 2 & 1200\\
B03 & wireless/OLSR & static & 10 & 10 & 1200\\
B04 & wireless/OLSR & static & 10 & 5 & 1200\\
B05 & wireless/OLSR & static & 10 & 2 & 1200\\
B06 & wireless/OLSR & static & 5 & 5 & 1200\\
B07 & wireless/OLSR & static & 5 & 2 & 1200\\
T00 & wired/Ethernet & static & 20 & 10 & 1100\\
T01 & wired/Ethernet & static & 20 & 5 & 900\\
T02 & wired/Ethernet & static & 20 & 2 & 1200\\
T03 & wired/Ethernet & static & 10 & 10 & 1200\\
T04 & wired/Ethernet & static & 10 & 5 & 1150\\
T05 & wired/Ethernet & static & 10 & 2 & 1200\\
T06 & wired/Ethernet & static & 5 & 5 & 1100\\
T07 & wired/Ethernet & static & 5 & 2 & 1000\\
C01 & wireless/OLSR & churn & 20 & 10 & 1200\\
C04 & wireless/OLSR & churn & 10 & 5 & 1200\\
C06 & wireless/OLSR & churn & 5 & 2 & 1200\\
D01 & wired/Ethernet & churn & 20 & 10 & 1200\\
D04 & wired/Ethernet & churn & 10 & 5 & 1200\\
D06 & wired/Ethernet & churn & 5 & 2 & 1200\\
\hline
\end{tabular}
\caption{Experiments and parameters conducted within this thesis}
\label{tab:experiments}
\end{table}
\subsection{Cycle}
All time dependent graphs in this section use a metric called cycle.A cycle
is defined as the time period during which every node in an experiment has initiated a shuffle request. Since each node initiates a shuffle once for
every shuffle time period a cycle equals the shuffle period time. In all
conducted experiments the shuffle time as set to 3000ms

\subsection{Static Scenario}
In the static scenario the experiments start with 300 nodes that do not die
until the end of the experiment. During the experiments the nodes exchange
neighbor information asynchronously at a fixed rate. The experiments were conducted in a wired and
wireless network to examine Cyclon's behaviour with different package loss
rates. Three different cache sizes and three different shuffle lengths have been
used and combined. The static scenario is used to
study topology properties and convergence behaviour of Cyclon. It
does not resemble typical usage of a Peer-to-Peer network, where peers join and leave frequently.  
\subsection{Churn Scenario}
In real Peer-to-Peer networks clients often join and leave the network
frequently. The rate in which the network changes due to this peer behaviour is
called churn rate. The churn scenario experiments are to simulate Cyclon in a
highly dynamic network. The churn mode much like the static mode starts with 300
peers, which then calculate a time to live. They behave the same as in the
static mode until they exceed their time to live value. The peers then stop all
communication. Each time a peer dies another Cyclon peer is started. This way
there is always up to 300 peers in the network. The time to live is a random
number between 100 and 300 of cycles with equal probabilities. Most of the
experiments run for 1200 cycles so that the minimum number of unique peers
within an experiment is 1200 and the maximum 3600. The number of simultaneously
running peers never exceeds 300. This scenario is used to obtain data on how
Cyclon behaves in a more realistic environment.


\section{Experiment Results}

\subsection{Connectivity}
In all experiments the initial topology after bootstrapping is a network with a
single weakly connected component. That means that each node is either known by
or knows about a node in the component. This is to ensure no node starts
without a connection to the network, since Cyclon does not have merging
capabilities. In all experiments conducted as described in table
\ref{tab:experiments} the components are examined after 1200 cycles (example
see figure \ref{fig:WeaklyCompB04}).\\

\begin{figure}
	\includegraphics[width=\textwidth,keepaspectratio]{./graphics/analysis/WeaklyComponentB04.pdf}
	\caption{Overlay Topology after 1200 cycles, experiment B04
	(static,wireless,cache size 10, shuffle length 5); All nodes are of the same
	color indicating a single weakly connected component.}
	\label{fig:WeaklyCompB04}
\end{figure}

The result in all static experiments is that at the end of the experiment the
network still consists out of a single weakly connected component. Since Cyclon does not have
merging capabilities other than the bootstrapping process, this means the
network does not split up but stays connected throughput the duration of the
experiment.\\
In all churn mode experiments the result after 1200 cycles is a little bit
different. There seem to be more than one weakly connected component. A closer
examination shows a big component and several components with one or two
nodes. The nodes within the single node components all revealed to be new nodes
before bootstrapping process. The nodes in in the two-node-components failed to
bootstrap to the main component. Instead they bootstrapped to each other
revealing a weakness in the bootstrapping process, but not in the Cyclon
protocol itself.\\
Summarizing the Cyclon shuffling algorithm seems not to divide a network once it
has been successfully bootstrapped, even under conditions like moderate package
loss and high churn.

\subsection{Degree Distribution}
We have now examined the overall connectivity of the network, which is a
constraint for a membership management overlay topology to work. Although the
connectivity of the network is crucial, the fact that there is only one
connected component, does say little about the topology of the network other
than that it is and stays connected. To examine the topologies weaknesses and
strengths the next part will show the degree distribution and how it evolves
during experiments. Since Cyclon produces a directed graph we will seperatly
examine the out-degree distribution and in-degree distribution.
\subsubsection{Out-Degree Distribution}
\label{subsubsec:outDegree}
Because Cyclon has a fixed cache size and it fills the cache up before deleting
any pointers from it, the expected out-degree of all nodes is to be the same or
smaller than the cache size. The out-degree distribution indicates therefore how
full the networks caches are. Some of the out-degree distribution results are
shown in figures \ref{fig:outDegreeDistT00}, \ref{fig:outDegreeDistB00}. \\

\begin{figure}
	\includegraphics[width=\textwidth,keepaspectratio]{./graphics/analysis/outDegreeGraphB00.pdf}
	\caption{Out-Degree Distribution in wireless Static Mode, cache size=20,
	shuffle length=10}
	\label{fig:outDegreeDistB00}
\end{figure}

\begin{figure}
	\includegraphics[width=\textwidth,keepaspectratio]{./graphics/analysis/outDegreeGraphT00.pdf}
	\caption{Out-Degree Distribution in wired Static Mode, cache size=20, shuffle
	length=10}
	\label{fig:outDegreeDistT00}
\end{figure}

Figure \ref{fig:outDegreeDistB00} shows the out-degree distribution in a
wireless static mode experiment with cache size and shuffle length of size 10
over time. The first probe is taken after 200 cycles and shows that almost all
nodes have an out-degree of 20 which was expected. What is not expected is that
there are some nodes with an out-degree higher than the cache size.\\ 
After examining the source code of the implementation, the effect can be
explained as follows. When a peer sends a shuffle request it tags the pointers
it sent. On reception of a response it first fills out available empty space in
the cache and then replaces tagged pointers with the new pointers from the
response. This ensures that a pointer is never lost unless a packet containing
it gets lost. The problem arises when a peer \emph{A} sends a shuffle request
to peer \emph{B}, tags the sent pointers in its cache and before receiving a
response from \emph{B}, receives a request from a different peer \emph{C}. It
creates a response for \emph{C} possibly containing pointers that are tagged
and replaces them with pointers from the request of \emph{C}. Up to now the
cache is within its bounds, but when the shuffle response from peer \emph{B}
arrives there is not enough tagged pointers fro replacement. Unfortunately this
effect was discovered too late to be prevented. Since it occurs rather rarely ,
e.g. up to 3 nodes from 300 in figure \ref{fig:outDegreeDistB00}, it has a
limited impact on the results.\\
Resuming the description of figure \ref{fig:outDegreeDistB00}, after 400 cycles
the out-degree of almost all nodes is between 16 and 20, which shows that some
peers delete parts of their cache not only to make space for new ones. This is
due to packet loss. When peer \emph{A} sends a shuffle request and does not
receive an answer, it deletes the pointer of the current target peer from its
cache. In static mode the reason for a missing response can only be loss of the
request or the response.\\
After 400 cycles the degree distribution does not change fundamentally; in the
following probes after 600,800,00 and 1200 cycles a tendency towards 19 as the
out-degree can be observed.\\
In figure \ref{fig:outDegreeDistT00} the -out-degree distribution of a wired
static mode experiment is shown for comparison. Throughout the experiment almost
all the nodes have an out-degree of 20. This shows that with virtually no packet
loss the peers do not delete pointers from their caches unless they exchange
them for new ones. There are again some peers which have more pointers than the
cache size would allow, which again is due to the special effect mentioned
above.

\begin{figure}
\begin{subfigure}{.5\textwidth}
	\includegraphics[width=\textwidth,keepaspectratio]{./graphics/analysis/outDegreeGraphB00.pdf}
	\caption{cache size=20, shuffle
	length=10}
	\label{fig:outDegreeDistB00small}
\end{subfigure}%
\begin{subfigure}{.5\textwidth}
	\includegraphics[width=\textwidth,keepaspectratio]{./graphics/analysis/outDegreeGraphB01.pdf}
	\caption{cache size=20, shuffle
	length=5}
	\label{fig:outDegreeDistB01small}
\end{subfigure}
\begin{subfigure}{.5\textwidth}
	\includegraphics[width=\textwidth,
	keepaspectratio]{./graphics/analysis/outDegreeGraphB02.pdf} \caption{cache size=20, shuffle length=2}
	\label{fig:outDegreeDistB02small}
\end{subfigure}%
\begin{subfigure}{.5\textwidth}
	\includegraphics[width=\textwidth,keepaspectratio]{./graphics/analysis/outDegreeGraphB03.pdf}
	\caption{cache size=10, shuffle
	length=10}
	\label{fig:outDegreeDistB03small}
\end{subfigure}
\begin{subfigure}{.5\textwidth}
	\includegraphics[width=\textwidth,keepaspectratio]{./graphics/analysis/outDegreeGraphB04.pdf}
	\caption{cache size=10, shuffle
	length=5}
	\label{fig:outDegreeDistB04small}
\end{subfigure}%
\begin{subfigure}{.5\textwidth}
	\includegraphics[width=\textwidth,keepaspectratio]{./graphics/analysis/outDegreeGraphB05.pdf}
	\caption{cache size=10, shuffle
	length=2}
	\label{fig:outDegreeDistB05small}
\end{subfigure}
\begin{subfigure}{.5\textwidth}
	\includegraphics[width=\textwidth,keepaspectratio]{./graphics/analysis/outDegreeGraphB06.pdf}
	\caption{cache size=5, shuffle
	length=5}
	\label{fig:outDegreeDistB06small}
\end{subfigure}%
\begin{subfigure}{.5\textwidth}
	\includegraphics[width=\textwidth,keepaspectratio]{./graphics/analysis/outDegreeGraphB07.pdf}
	\caption{cache size=5, shuffle
	length=2}
	\label{fig:outDegreeDistB07small}
\end{subfigure}
\caption{Out-Degree Distribution in wireless static scenario}
\end{figure}

\begin{figure}
\begin{subfigure}{.5\textwidth}
	\includegraphics[width=\textwidth,keepaspectratio]{./graphics/analysis/outDegreeGraphT00.pdf}
	\caption{cache size=20, shuffle
	length=10}
	\label{fig:outDegreeDistT00small}
\end{subfigure}%
\begin{subfigure}{.5\textwidth}
	\includegraphics[width=\textwidth,keepaspectratio]{./graphics/analysis/outDegreeGraphT01.pdf}
	\caption{cache size=20, shuffle
	length=5}
	\label{fig:outDegreeDistT01small}
\end{subfigure}
\begin{subfigure}{.5\textwidth}
	\includegraphics[width=\textwidth,keepaspectratio]{./graphics/analysis/outDegreeGraphT02.pdf}
	\caption{cache size=20, shuffle
	length=2}
	\label{fig:outDegreeDistT02small}
\end{subfigure}%
\begin{subfigure}{.5\textwidth}
	\includegraphics[width=\textwidth,keepaspectratio]{./graphics/analysis/outDegreeGraphT03.pdf}
	\caption{cache size=10, shuffle
	length=10}
	\label{fig:outDegreeDistT03small}
\end{subfigure}
\begin{subfigure}{.5\textwidth}
	\includegraphics[width=\textwidth,keepaspectratio]{./graphics/analysis/outDegreeGraphT04.pdf}
	\caption{cache size=10, shuffle
	length=5}
	\label{fig:outDegreeDistT04small}
\end{subfigure}%
\begin{subfigure}{.5\textwidth}
	\includegraphics[width=\textwidth,keepaspectratio]{./graphics/analysis/outDegreeGraphT05.pdf}
	\caption{cache size=10, shuffle
	length=2}
	\label{fig:outDegreeDistT05small}
\end{subfigure}
\begin{subfigure}{.5\textwidth}
	\includegraphics[width=\textwidth,keepaspectratio]{./graphics/analysis/outDegreeGraphT06.pdf}
	\caption{cache size=5, shuffle
	length=5}
	\label{fig:outDegreeDistT06small}
\end{subfigure}%
\begin{subfigure}{.5\textwidth}
	\includegraphics[width=\textwidth,keepaspectratio]{./graphics/analysis/outDegreeGraphT07.pdf}
	\caption{cache size=5, shuffle
	length=2}
	\label{fig:outDegreeDistT07small}
\end{subfigure}
\caption{Out-Degree Distribution in wired static scenario}
\end{figure}

\subsubsection{In-Degree Distribution}
\label{subsubsec:inDegree}
The in-degree of a node expresses the amount of nodes in the network having a
pointer on that node in their caches. In other words it shows the popularity of
a node. The goal of Cyclon is to have an even in-degree distribution, so that
every node is as good represented in the accumulated network cache as
possible.\\

\begin{figure}
	\includegraphics[width=\textwidth,keepaspectratio]{./graphics/analysis/degreeGraphB05.pdf}
	\caption{In-Degree Distribution in wireless static mode, cache size=10, shuffle
	length=2}
	\label{fig:inDegreeDistB05}
\end{figure}

As an example of how the in-degree distribution changes in time, figure
\ref{fig:inDegreeDistB05} of the wireless static experiment with cache size 10 and shuffle
length 2 will be discussed. After 200 cycles most nodes have a small in-degree,
but there are some with a very high one. Following the time line after 400 and
600 cycles the number of nodes with a very high in-degree dwindles, while most
nodes have an in-degree around 10. In comparison the in-degree of figures
\ref{fig:inDegreeDistB03small} and \ref{fig:inDegreeDistB04small} do not
distribute so evenly. The difference between these experiments is the shuffle
length. The same can be observed in wired experiments (see figures
\ref{fig:inDegreeDistT03small},
\ref{fig:inDegreeDistT04small}, \ref{fig:inDegreeDistT05small}). It seems that
there is a connection between the shuffle length and the smoothing in in-degree distribution.\\
There is two possibilities to change an in-degree through shuffling. The first
is when a node \emph{A} is chosen by another node \emph{B} as the next shuffle
target from its neighbor cache. If node \emph{B}'s cache has been full before
the shuffling, then node \emph{A} will no longer be in the cache of \emph{B}
after the shuffling. Node \emph{A} will now have a new pointer to node \emph{B}.
This means that the in-degree of \emph{B} is decremented by one and the
in-degree of \emph{A} is incremented by one. In this case a higher in-degree leads to a higher probability of being chosen as a shuffle target, which leads to a higher
probability to have the in-degree decremented. Since a
low in-degree means that the probability to be chosen as a shuffle target is low, also lowers the
probability to have the in-degree decremented.The probability $P(X_A)$ for a
node \emph{X} to be node \emph{A} in this scenario is dependent solely on the
in-degree $d$ and the cache size $c$ of the nodes in the network. During a cycle
a node \emph{X} is exactly one time node \emph{B} in this scenario. The
probability $P(X_B)$ therefore is independent from
shuffle length and in-degree. \[P(X_A) = \frac{d}{c}\]
The second way to change the in-degree is when node \emph{A} requests a shuffle
with node \emph{B} and both have node \emph{C} in their cache. If node \emph{A}
chooses \emph{C} to include in the shuffle request and node \emph{B} does not
chose node \emph{C} to put in the response the in-degree of \emph{C} is lowered
by one. Instead one of the nodes node \emph{B} chose to put in the response to
node \emph{A}, let it be node \emph{D}, has now an in-degree incremented by one.
In this scenario a high in-degree of node \emph{C} means there is a high
probability that two exchanging nodes (\emph{A} and \emph{B}) both have node
\emph{C} in their cache resulting in a higher probability to have the in-degree
decrement. But in this scenario a higher in-degree also means a higher chance to
be node \emph{D} which results in having an even higher in-degree. The chance
for a node to be node \emph{C} in this scenario is lower then the chance to be
node \emph{D}. This can be shown by a simple formula. Let there be a set $N$ of
$n$ nodes of which $d\leq n$ nodes have a pointer on node \emph{X}. Let $X_C$ be
the event that Node \emph{X} is of the same type as node \emph{C} in the
scenario above and $X_D$ the event that \emph{X} is of the same type as
\emph{D}.
Let $l$ be the shuffle length and $c$ the cache size.
The probabilities for each $X_1$ and $X_2$ are:
\[p(X_C) =
\overbrace{\frac{d^2}{n(n-1)}}^{\text{prob. of node A and B having a pointer on
X}} \cdot\underbrace{\frac{l}{c}(1-\frac{l}{c})}_{\text{prob. of X 
being chosen for sending by only one of the nodes}} \]
\[p(X_D) =
\overbrace{\frac{d(n-d)}{n(n-1)}}^{\text{prob. of only one node having
pointer on X}}
\cdot\underbrace{\frac{l}{c}}_{\text{prob. of X being chosen for sending}}\]

With the above the following equation can be given for the probable change of
the in-degree of a node during one cycle. This assumes that there is no packet
loss or latency in the network and that all caches are full.
\begin{align*}
	d_{new} &= p(X_B) - p(X_A) - p(X_C) + p(X_D) \\
	&= 1 - \frac{d}{c} - \frac{l\cdot d^2}{c\cdot n(n-1)}(1-\frac{l}{c}) 
+ \frac{l\cdot d(n-d)}{c\cdot n(n-1)} \\
	&= 1 - \frac{d}{c} - \frac{dl}{cn(n-1)}\left(\frac{dl}{c}+ n \right)
\end{align*}
For small n this means that the second scenario has an impact on how the
in-degree changes during a cycle. This shows why the shuffle length to cache
size ratio has such a big impact on network development in small networks. When
on the other hand the network gets bigger and $l,c,d$ are constant, then it is easy
to see that the impact of the shuffle to cache size ratio dwindles with growing
number of nodes:

\begin{align*}
\lim_{n \to \infty} d_{new} = 1 - \frac{d}{c}
\end{align*}

The first scenario smooths the in-degrees of all nodes and the second
scenario does the opposite. This proves that the strange behaviour of the
in-degree distribution is showing in the experiment due to the small number of
total nodes. In a bigger network the number of nodes $n$ makes the
probabilities $P(X_C)$ and $P(X_D)$ very small while the probabilities $P(X_A)$
and $P(X_B)$ remain constant.

\begin{figure}
\begin{subfigure}{.5\textwidth}
	\includegraphics[width=\textwidth,keepaspectratio]{./graphics/analysis/degreeGraphB00.pdf}
	\caption{cache size=20, shuffle
	length=10}
	\label{fig:inDegreeDistB00small}
\end{subfigure}%
\begin{subfigure}{.5\textwidth}
	\includegraphics[width=\textwidth,keepaspectratio]{./graphics/analysis/degreeGraphB01.pdf}
	\caption{cache size=20, shuffle
	length=5}
	\label{fig:inDegreeDistB01small}
\end{subfigure}
\begin{subfigure}{.5\textwidth}
	\includegraphics[width=\textwidth,keepaspectratio]{./graphics/analysis/degreeGraphB02.pdf}
	\caption{cache size=20, shuffle
	length=2}
	\label{fig:inDegreeDistB02small}
\end{subfigure}%
\begin{subfigure}{.5\textwidth}
	\includegraphics[width=\textwidth,keepaspectratio]{./graphics/analysis/degreeGraphB03.pdf}
	\caption{cache size=10, shuffle
	length=10}
	\label{fig:inDegreeDistB03small}
\end{subfigure}
\begin{subfigure}{.5\textwidth}
	\includegraphics[width=\textwidth,keepaspectratio]{./graphics/analysis/degreeGraphB04.pdf}
	\caption{cache size=10, shuffle
	length=5}
	\label{fig:inDegreeDistB04small}
\end{subfigure}%
\begin{subfigure}{.5\textwidth}
	\includegraphics[width=\textwidth,keepaspectratio]{./graphics/analysis/degreeGraphB05.pdf}
	\caption{cache size=10, shuffle
	length=2}
	\label{fig:inDegreeDistB05small}
\end{subfigure}
\begin{subfigure}{.5\textwidth}
	\includegraphics[width=\textwidth,keepaspectratio]{./graphics/analysis/degreeGraphB06.pdf}
	\caption{cache size=5, shuffle
	length=5}
	\label{fig:inDegreeDistB06small}
\end{subfigure}%
\begin{subfigure}{.5\textwidth}
	\includegraphics[width=\textwidth,keepaspectratio]{./graphics/analysis/degreeGraphB07.pdf}
	\caption{cache size=5, shuffle
	length=2}
	\label{fig:inDegreeDistB07small}
\end{subfigure}
\caption{In-Degree Distribution in wireless static scenario}
\end{figure}

\begin{figure}
\begin{subfigure}{.5\textwidth}
	\includegraphics[width=\textwidth,keepaspectratio]{./graphics/analysis/degreeGraphT00.pdf}
	\caption{cache size=20, shuffle
	length=10}
	\label{fig:inDegreeDistT00small}
\end{subfigure}%
\begin{subfigure}{.5\textwidth}
	\includegraphics[width=\textwidth,keepaspectratio]{./graphics/analysis/degreeGraphT01.pdf}
	\caption{cache size=20, shuffle
	length=5}
	\label{fig:inDegreeDistT01small}
\end{subfigure}
\begin{subfigure}{.5\textwidth}
	\includegraphics[width=\textwidth,keepaspectratio]{./graphics/analysis/degreeGraphT02.pdf}
	\caption{cache size=20, shuffle
	length=2}
	\label{fig:inDegreeDistT02small}
\end{subfigure}%
\begin{subfigure}{.5\textwidth}
	\includegraphics[width=\textwidth,keepaspectratio]{./graphics/analysis/degreeGraphT03.pdf}
	\caption{cache size=10, shuffle
	length=10}
	\label{fig:inDegreeDistT03small}
\end{subfigure}
\begin{subfigure}{.5\textwidth}
	\includegraphics[width=\textwidth,keepaspectratio]{./graphics/analysis/degreeGraphT04.pdf}
	\caption{cache size=10, shuffle
	length=5}
	\label{fig:inDegreeDistT04small}
\end{subfigure}%
\begin{subfigure}{.5\textwidth}
	\includegraphics[width=\textwidth,keepaspectratio]{./graphics/analysis/degreeGraphT05.pdf}
	\caption{cache size=10, shuffle
	length=2}
	\label{fig:inDegreeDistT05small}
\end{subfigure}
\begin{subfigure}{.5\textwidth}
	\includegraphics[width=\textwidth,keepaspectratio]{./graphics/analysis/degreeGraphT06.pdf}
	\caption{cache size=5, shuffle
	length=5}
	\label{fig:inDegreeDistT06small}
\end{subfigure}%
\begin{subfigure}{.5\textwidth}
	\includegraphics[width=\textwidth,keepaspectratio]{./graphics/analysis/degreeGraphT07.pdf}
	\caption{cache size=5, shuffle
	length=2}
	\label{fig:inDegreeDistT07small}
\end{subfigure}
\caption{In-Degree Distribution in wired static scenario}
\end{figure}

\FloatBarrier
\subsection{Cluster Coefficient}
The clustering coefficient is another indicator for complex graphs. It
calculates how many neighbors of each node are neighbors themselves. In other
words the higher the cluster coefficient of a node is, the more complete the
cluster around that node. The average cluster coefficient in a network is the
mean value of the local cluster coefficient of all networks. A high cluster
coefficient indicates a network with strong cluster building. If there is a
limited amount of edges, as is in Cyclon a high cluster coefficient also means
that the clusters are weakly connected outside of a cluster. This can be
explained as follows. A node has a fixed number of edges. If most of those edges
go into its own cluster then there is not many edges left to go outside of the
cluster. Same applies to all other nodes in the cluster. Weakly connected
clusters also mean that the links between those clusters are the weak points of
the whole network. Therefore for Cyclon it is desirable to have a low
average cluster coefficient. 
In figure \ref{fig:ClusterCoef10small} the average
cluster coefficients for wireless static experiments with cache size 10 are
shown. As a reference the average cluster coefficient of a random graph ensemble
with the same amount of nodes and edges is also projected into the figure. The
values after 400 cycles show to be stable. In this sample the the value of the
experiment with shuffle length 2 is about twice the reference value of the
random graph. The difference to the ideal can not be explained completely with
packet loss. Figure \ref{fig:ClusterCoefNWI} shows a comparison between
wired and wireless experiments. Wired experiments are virtually lossless but still are
far away from being close to the ideal reference line. The abrupt end of some of
the graphs in \ref{fig:ClusterCoefNWI} is resulting from a premature end of
the experiment due to technical reasons. It does not hinder from seeing that the
average cluster coefficients of the lossless experiments are generally lower
than the others. An exception might be the runs with cache size 20, and indeed,
as we can see in figure \ref{fig:ClusterCoef20small}, the cluster coefficient is
generally closer to the ideal for graphs with cache size 20.

\begin{figure}
	\centering
	\includegraphics[width=0.6501\textwidth,keepaspectratio]{./graphics/analysis/clusterGraphNWI.pdf}
	\caption{Cluster Coefficient in wireless and wired static mode}
	\label{fig:ClusterCoefNWI}
\end{figure}

\begin{figure}
\centering
\begin{subfigure}{\textwidth}
	\centering
	\includegraphics[height=.34\textheight,keepaspectratio]{./graphics/analysis/clusterGraph5.pdf}
	\caption{ cache size=5}
	\label{fig:ClusterCoef5small}
\end{subfigure}
\begin{subfigure}{\textwidth}
	\centering
	\includegraphics[height=.34\textheight,keepaspectratio]{./graphics/analysis/clusterGraph10.pdf}
	\caption{ cache size=10}
	\label{fig:ClusterCoef10small}
\end{subfigure}
\begin{subfigure}{\textwidth}
	\centering
	\includegraphics[height=.34\textheight,keepaspectratio]{./graphics/analysis/clusterGraph20.pdf}
	\caption{cache size=20}
	\label{fig:ClusterCoef20small}
\end{subfigure}
\caption{Cluster coefficient in wireless static mode by cache size}
\end{figure}

\begin{figure}
\begin{subfigure}{\textwidth}
	\centering
	\includegraphics[height=.34\textheight,keepaspectratio]{./graphics/analysis/clusterGraphSL2.pdf}
	\caption{shuffle length=2}
	\label{fig:ClusterCoefSL2small}
\end{subfigure}
\begin{subfigure}{\textwidth}
	\centering
	\includegraphics[height=.34\textheight,keepaspectratio]{./graphics/analysis/clusterGraphSL5.pdf}
	\caption{shuffle length=5}
	\label{fig:ClusterCoefSL5small}
\end{subfigure}
\begin{subfigure}{\textwidth}
	\centering
	\includegraphics[height=.34\textheight,keepaspectratio]{./graphics/analysis/clusterGraphSL10.pdf}
	\caption{shuffle length=10}
	\label{fig:ClusterCoefSL10small}
\end{subfigure}
\caption{Cluster Coefficient in wireless static scenario by shuffle length}
\end{figure}

\begin{figure}
\begin{subfigure}{\textwidth}
	\centering
	\includegraphics[height=.34\textheight,keepaspectratio]{./graphics/analysis/clusterGraphChurn5.pdf}
	\caption{cache size 5}
	\label{fig:ClusterCoefChurn5small}
\end{subfigure}
\begin{subfigure}{\textwidth}
	\centering
	\includegraphics[height=.34\textheight,keepaspectratio]{./graphics/analysis/clusterGraphChurn10.pdf}
	\caption{cache size 10}
	\label{fig:ClusterCoefChurn5small}
\end{subfigure}
\begin{subfigure}{\textwidth}
	\centering
	\includegraphics[height=.34\textheight,keepaspectratio]{./graphics/analysis/clusterGraphChurn20.pdf}
	\caption{cache size 20}
	\label{fig:ClusterCoefChurn5small}
\end{subfigure}
\caption{Cluster Coefficient in wireless static scenario by shuffle length}
\end{figure}


\begin{figure}
\begin{subfigure}{.5\textwidth}

\end{subfigure}
\end{figure}
\FloatBarrier
\subsection{Path lengths}
\subsubsection{Diameter}
\label{subsubsec:diameter}
The diameter of a graph is the longest of all shortest paths within the network.
In other words it shows the maximum number of hops a broadcast message must be
send over to make sure it reaches every node. This is a important metric for
search and find operations on the network. The results of all experiments in
static mode are shown in table \ref{tab:diameter}. The results is the
diameter of the network topology after 1200 cycles if not indicated
differently.\\
The diameter for almost all static experiments is between 6-7. Exceptions are
the experiments where cache size is 5 and shuffle length is 2. This effect
is due to a combination of the overall smaller in-degree in experiments with
shuffle length 2 discussed in section \ref{subsubsec:inDegree} and the small
cache size. In networks containing nodes with a high in-degree those nodes take
the role of shortcuts as opposed to networks without such nodes. In combination
with the small cache size of 5 which restricts the number of overall edges in
the network, this leads to a high diameter.  


\begin{table}
\begin{tabular} { |p{1.9cm}|l|p{1.9cm}|p{2cm}|p{2.4cm}|p{1.9cm}|}
\hline
Experiment Name & Network Mode & Experiment Scenario & Cache Size (neighbors) &
Shuffle length (neighbors) & Diameter 1200 cycles \\
\hline \hline
B00 & wireless/OLSR & static & 20 & 10 & 6\\
B01 & wireless/OLSR & static & 20 & 5 & 7\\
B02 & wireless/OLSR & static & 20 & 2 & 6\\
B03 & wireless/OLSR & static & 10 & 10 & 7\\
B04 & wireless/OLSR & static & 10 & 5 & 6\\
B05 & wireless/OLSR & static & 10 & 2 & 7\\
B06 & wireless/OLSR & static & 5 & 5 & 7\\
B07 & wireless/OLSR & static & 5 & 2 & 10\\
T00 & wired/Ethernet & static & 20 & 10 & 6 (1100)\\
T01 & wired/Ethernet & static & 20 & 5 & 6 (900)\\
T02 & wired/Ethernet & static & 20 & 2 & 5\\
T03 & wired/Ethernet & static & 10 & 10 & 5\\
T04 & wired/Ethernet & static & 10 & 5 & 7 (1150)\\
T05 & wired/Ethernet & static & 10 & 2 & 7\\
T06 & wired/Ethernet & static & 5 & 5 & 6 (1100)\\
T07 & wired/Ethernet & static & 5 & 2 & 10 (1000)\\
\hline
\end{tabular}
\caption{Diameter after 1200 cycles if not specified differently}
\label{tab:diameter}
\end{table}

\subsubsection{Betweenness Centrality}
Betweenness centrality is a network metric to indicate how central a node is
based on the shortest paths running through it. For this the shortest paths of
all pairs of nodes are calculated. For each node the number of all occurrences
within the shortest paths is summed up. The resulting number shows how many
shortest paths of the network use this node. In this thesis the results are
additionally normalized, so that 0 means none of the shortest paths use this
node and 1 means all of the shortest paths use this node. In Peer-to-Peer
networks nodes with a high betweenness centrality will probably face more
traffic since they are on the shortest paths of more node pairs. This means that
the lower and more similar the betweenness centrality of all nodes in a network
is, the better the load balancing in the network will probably be. For sake of
comparison the ideal load balancing would be if every node had the same load to
bear.This will be used as a reference value, although this does not mean that
there is such a topology. It should also be stated that the absolute total
traffic in a network calculated like this, also depends on the actual length of
the shortest paths. This is to say that the betweenness centrality can only
indicate the load balancing and not the efficiency of the network. \[\frac{100\%
\text{ traffic}}{300 \text{ nodes}} \approx 0.33\% \text{ traffic per node}\]\\

\begin{figure}
	\includegraphics[width=\textwidth,keepaspectratio]{./graphics/analysis/BetweennessCentralityB06.pdf}
	\caption{Betweenness Centrality, wireless static mode, cache size 5, shuffle
	length 5}
	\label{fig:betweennessB06}
\end{figure}

As an example in figure \ref{fig:betweennessB06} the betweenness centrality of a
wireless static experiment with cache size and shuffle length 5 is shown. The
columns accumulate the number of nodes of which the betweenness centrality is
between the left and the right boundary of the column. In this figure it is
shown that the betweenness centrality of the nodes varies between 0 or close to
0 and 0.035. This means that if the assumed traffic caused by every node is the
same, some nodes have to bear up to 3.5\% of the network traffic. This value is
about 10 times higher as the ideal. At the same time more than two thirds of
the nodes have an betweenness centrality of 0 or close to it. In the assumed
traffic scenario this would mean no or very low traffic load. After examining
the specific nodes with a high betweenness centrality in this experiment, the
nodes turn out to be nodes that have also a high in-degree.
This explains the phenomenon, as those nodes act as a sort of shortcut in the
network an are therefore more probable to be on shortest paths between nodes.
This can again be explained by the small size of the network in combination with
the shuffle length.\\

\begin{figure}
	\includegraphics[width=\textwidth,keepaspectratio]{./graphics/analysis/BetweennessCentralityB07.pdf}
	\caption{Betweenness Centrality, wireless static mode, cache size 5, shuffle
	length 2}
	\label{fig:betweennessB07}
\end{figure}


For a betweenness centrality distribution of an experiment without
very high in-degree nodes, figure \ref{fig:betweennessB07} will be discussed
next. The experiment parameters are given with a shuffle length of 2 a cahce
size of 5 and the mode is wireless and static. As depicted the betweenness
centrality varies between 0 and about 0.045 which in the assumed traffic
scenario means that there are nodes bearing up to 4.5\% of network traffic. On
the other hand the number of nodes with a betweenness centrality of 0 or close
to 0 is less then half of the network size. Again in the assumed traffic load
scenario this would mean less than half of the nodes would have none or near to
none traffic load. \\

\begin{figure}
	\includegraphics[width=\textwidth,keepaspectratio]{./graphics/analysis/BetweennessCentralityB00.pdf}
	\caption{Betweenness Centrality, wireless static mode, cache size 20, shuffle
	length 10}
	\label{fig:betweennessB00}
\end{figure}

\begin{figure}
	\includegraphics[width=\textwidth,keepaspectratio]{./graphics/analysis/BetweennessCentralityB02.pdf}
	\caption{Betweenness Centrality, wireless static mode, cache size 20, shuffle
	length 2}
	\label{fig:betweennessB02}
\end{figure}

In figures \ref{fig:betweennessB00} and \ref{fig:betweennessB02} the difference
is even clearer. In figure \ref{fig:betweennessB00} the the range of betweenness
centrality goes from 0 to about 0.025, while almost two thirds of the nodes
have a value of 0 and close to 0. In figure \ref{fig:betweennessB02} the range
of betweenness centrality goes from 0 to about 0.02, while the number of nodes
with a value 0 or very close to 0 is only at one tenth of the network. Although
there are still nodes with a relatively high betweenness centrality the there
are a lot more nodes closer to the assumed ideal value of 0.003

\begin{figure}
\begin{subfigure}{.5\textwidth}
	\includegraphics[width=\textwidth,keepaspectratio]{./graphics/analysis/BetweennessCentralityB00.pdf}
	\caption{cache size=20, shuffle
	length=10}
	\label{fig:betweennessCentralityB00small}
\end{subfigure}%
\begin{subfigure}{.5\textwidth}
	\includegraphics[width=\textwidth,keepaspectratio]{./graphics/analysis/BetweennessCentralityB01.pdf}
	\caption{cache size=20, shuffle
	length=5}
	\label{fig:betweennessCentralityB01small}
\end{subfigure}
\begin{subfigure}{.5\textwidth}
	\includegraphics[width=\textwidth,keepaspectratio]{./graphics/analysis/BetweennessCentralityB02.pdf}
	\caption{cache size=20, shuffle
	length=2}
	\label{fig:betweennessCentralityB02small}
\end{subfigure}%
\begin{subfigure}{.5\textwidth}
	\includegraphics[width=\textwidth,keepaspectratio]{./graphics/analysis/BetweennessCentralityB03.pdf}
	\caption{cache size=10, shuffle
	length=10}
	\label{fig:betweennessCentralityB03small}
\end{subfigure}
\begin{subfigure}{.5\textwidth}
	\includegraphics[width=\textwidth,keepaspectratio]{./graphics/analysis/BetweennessCentralityB04.pdf}
	\caption{cache size=10, shuffle
	length=5}
	\label{fig:betweennessCentralityB04small}
\end{subfigure}%
\begin{subfigure}{.5\textwidth}
	\includegraphics[width=\textwidth,keepaspectratio]{./graphics/analysis/BetweennessCentralityB05.pdf}
	\caption{cache size=10, shuffle
	length=2}
	\label{fig:betweennessCentralityB05small}
\end{subfigure}
\begin{subfigure}{.5\textwidth}
	\includegraphics[width=\textwidth,keepaspectratio]{./graphics/analysis/BetweennessCentralityB06.pdf}
	\caption{cache size=5, shuffle
	length=5}
	\label{fig:betweennessCentralityB06small}
\end{subfigure}%
\begin{subfigure}{.5\textwidth}
	\includegraphics[width=\textwidth,keepaspectratio]{./graphics/analysis/BetweennessCentralityB07.pdf}
	\caption{cache size=5, shuffle
	length=2}
	\label{fig:betweennessCentralityB07small}
\end{subfigure}
\caption{Betweenness Centrality in wireless static mode}
\end{figure}



\begin{figure}
\begin{subfigure}{.5\textwidth}
	\includegraphics[width=\textwidth,keepaspectratio]{./graphics/analysis/BetweennessCentralityT00.pdf}
	\caption{cache size=20, shuffle
	length=10}
	\label{fig:betweennessCentralityT00small}
\end{subfigure}%
\begin{subfigure}{.5\textwidth}
	\includegraphics[width=\textwidth,keepaspectratio]{./graphics/analysis/BetweennessCentralityT01.pdf}
	\caption{cache size=20, shuffle
	length=5}
	\label{fig:betweennessCentralityT01small}
\end{subfigure}
\begin{subfigure}{.5\textwidth}
	\includegraphics[width=\textwidth,keepaspectratio]{./graphics/analysis/BetweennessCentralityT02.pdf}
	\caption{cache size=20, shuffle
	length=2}
	\label{fig:betweennessCentralityT02small}
\end{subfigure}%
\begin{subfigure}{.5\textwidth}
	\includegraphics[width=\textwidth,keepaspectratio]{./graphics/analysis/BetweennessCentralityT03.pdf}
	\caption{cache size=10, shuffle
	length=10}
	\label{fig:betweennessCentralityT03small}
\end{subfigure}
\begin{subfigure}{.5\textwidth}
	\includegraphics[width=\textwidth,keepaspectratio]{./graphics/analysis/BetweennessCentralityT04.pdf}
	\caption{cache size=10, shuffle
	length=5}
	\label{fig:betweennessCentralityT04small}
\end{subfigure}%
\begin{subfigure}{.5\textwidth}
	\includegraphics[width=\textwidth,keepaspectratio]{./graphics/analysis/BetweennessCentralityT05.pdf}
	\caption{cache size=10, shuffle length=2}
	\label{fig:betweennessCentralityT05small}
\end{subfigure}
\begin{subfigure}{.5\textwidth}
	\includegraphics[width=\textwidth,keepaspectratio]{./graphics/analysis/BetweennessCentralityT06.pdf}
	\caption{cache size=5, shuffle
	length=5}
	\label{fig:betweennessCentralityT06small}
\end{subfigure}%
\begin{subfigure}{.5\textwidth}
	\includegraphics[width=\textwidth,keepaspectratio]{./graphics/analysis/BetweennessCentralityT07.pdf}
	\caption{cache size=5, shuffle
	length=2}
	\label{fig:betweennessCentralityT07small}
\end{subfigure}
\caption{Betweenness Centrality in wired static mode}
\end{figure}

\FloatBarrier
\section{Summary}
	This section contains a summary of the results discussed above.
	The experiments have shown that Cyclon provides an overlay topology that does
	not loose connectivity once established even with moderate package loss and a
	high churn of clients. They have also shown that the topology metrics converge
	to certain values. The typical out-degree in an overlay topology created by
	Cyclon has shown to be close to the cache size. Although with moderate package
	loss the mean out-degree is smaller than without, it is a small difference.
	Most in-degrees in a topology have shown to be close to the cache size if the
	shuffle length was small enough. The experiments node count has shown to be to
	small to show the same for bigger shuffle lengths. The average cluster
	coefficients of all experiments have not been bigger than twice the average
	cluster coefficients of comparable random graphs. Packet loss seems to also
	have a small negative impact on cluster coefficients. The diameter of the
	network is similar to those of small world networks. The betweenness
	centrality for a sufficiently small shuffle length shows that Cyclon does not
	create super nodes, that would be weak points in topology structure. \\
	
	The number of nodes in the experiment has shown to be too small for most of the
	experiment parameters. Especially the shuffle length to cache size ratio has
	proven to have a big effect on in-degree distribution in disfavor of the
	networks symmetry. The networks latency also has shown to have an impact on
	this implementation, in the way that in rare cases the cache size is above its
	initial boundary. Nodes with high in-degree are especially prone to this
	behaviour. The effects of network behaviour will be described in the next
	subsection
	
	
\subsection{Impacts of Package Loss on Topology Development}
	The goal of the experiments was to find what impact package loss has on the
	topology. This is why a wireless network using UDP packets was chosen. The
	shuffle response packet and the shuffle request packet both can get lost. If
	the latter gets lost this means that in this cycle the node does not make
	a shuffle exchange and that the target node has one less response to send.
	For the whole network this means that there is one less shuffle in the
	network. This slows network development, which has an impact on how fast the
	network converges and repairs itself after big failures.
	If the shuffle response gets lost the impact is larger. The target node has
	exchanged its nodes with the nodes from the shuffle request, but the request
	node cannot. This leads to a change in in-degree distribution in favor of the
	nodes from the shuffle request packet. In extreme cases where nodes are only
	known by one other node and are the only connection to a different part of
	the network , this could lead to a network partitioning. In the experiments
	conducted for this thesis the statistics server counted up to 10\% lost
	response packages. As stated in the diameter section \ref{subsubsec:diameter}
	of this chapter, no partitionings have been observed.\\
	Lost response packets also change the cluster coefficient of the responding
	node. This is because after the node exchanges their neighbors to the ones from
	the request packet, it has more neighbors in common with the requesting node,
	which is now also its neighbor. The negative impact on network metric values of
	this scenario also is dependent on the shuffle length to cache size ratio. The
	higher the ratio the more negative the impact. When the shuffle length is the
	same as the cache size and the caches are full in both nodes and a response
	package gets lost the nodes have almost identical caches. Summerizing a
	moderate paket loss has negative impact on the networks balance but in the
	experiments did not produce a fatal error.
\subsection{Impacts of Latency on Topology Development}
	Network latencies on the DES-Testbed are very small compared to the socket
	timeout set in this implementation of Cyclon, but still have an impact on the
	networks behaviour. As described in the section above and the section
	\ref{subsubsec:outDegree} the latency produces an error in cache size
	management. The problem arisis when a node sends a shuffling request and before
	receiving a response gets another shuffle request. There are several
	possibilities to change Cyclons behaviour in that case. One would be to simply drop
	the package. This leads directly to a higher request package loss ratio.
	Another possibility is to wait for the response before processing the new
	request. While this works fine for the first node it could possibly let the
	timeout of the second requesting node to run out before receiving its response. This
	solution could lead to a higher response package loss. Another option would be
	for the first node to pick only pointers from the untagged portion of its cache
	to send in the response packet. Depending on the shuffle length to cache size
	ratio this could mean an only partially filled or even empty response packet.
	On receiving an only partially filled package the second requesting node would
	have to know which nodes it can replace otherwise pointers could get lost.
	
	
	
	
	
	
	
	