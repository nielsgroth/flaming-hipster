\chapter{Properties and Requirements of Gossip-Based Overlay Topologies} 
\label{chap:properties}

\section{General Requirements}
There are a few general requirements a distributed overlay topology building
solution should embrace. They are presented in the following section.
\subsubsection{Simplicity}
A solution for a distributed overlay topology building should be simple in its
deployment and lightweight in its operation. 
\subsubsection{Scalability}
Due to the large numbers of possible participants the solution should be
designed to work with any number of nodes. This means mainly that the solution
is constrained by the finite resources of a single node while the size of the
network varies and is virtually infinite.

\subsubsection{Symmetry}
In a distributed solution all participants should have equal roles and work in
the same way. This reduces the risk for single point failures and contributes to
the simplicity of the solution.

\section{Functional Requirements}
While the general requirements consider design qualities of the Protocol,
functional requirements concern the resulting network topology and it's
properties. They describe the resulting network and it’s usefulness with global
properties, and local node specific properties.

\subsection{Global Properties}

\subsubsection{Connectivity}
The connectivity describes a network graphs property to interconnect the
participants with each other. It can be indicated by the number of network
components or the size of the largest component if there is more then one. It is
desired to have all the participants connected to solely one component, but if
this can't be reached the size of the biggest component can be used as a quality
indicator. In this thesis the connectivity will be measured as a binary
property, saying if the nodes are completely connected or not in different
scenarios.
\subsubsection{Convergence}
Convergence is a time dependent property which indicates if a network converges
or not. It is the answer to the question if the networks other properties
converge to a certain value or if they jump arbitrarily through the scale.
In this thesis it is well desired for a network to converge, for it is close to
impossible to examine arbitrarily behaving networks.

\subsubsection{Diameter}
The diameter of a network graph is the longest shortest path between any two
nodes within the network. The higher the number is, the longer or costlier
communication between those two nodes is. This number is especially important if
the network uses flooding since it determines the number of hops such a flooding
message has to take to reach all nodes in the network. 

\subsubsection{Average Clustering Coefficient}
The average clustering coefficient is, as the name indicates,  the average of
all local clustering coefficients in a network. It describes the overall
clustering in a network and a high value indicates a network topology which has
a good interconnection within neighbourhoods while having few links between them.
A low value on the other hand indicates a highly distributed interconnection
between all nodes and a low number of highly interconnected neighbourhoods. A low
value is preferred for global information dissemination and shows better self
healing capabilities after major node failures.
\subsection{Local Properties}
When considering properties of single nodes within the observed network, one
speaks of local properties. They are specific to a single node and can be
accumulated with various methods. The most common would be as described above
the average, median, maximum or minimum calculation. 
\subsubsection{Degree Distribution}
The degree distribution is another way to describe the quality of
interconnection within a network graph. It counts the ingoing and outgoing edges
of a node in a graph. While Cyclon's functionality limits the number of outgoing
edges to a finite maximum it does not so with the ingoing ones. The number of
ingoing edges can indicate the work load a node has to process. It is clear that
a node which has significantly more ingoing edges than another will be more
probably messaged to then the other. This would lead to a higher work load and
is not compatible to the general symmetry requirement. The goal should then be
to reach an even indegree distribution over all the nodes in the network to
distribute the work load as equal as possible. The outdegree will also be
considered although it seems clear that the algorithm of Cyclon tries to fill
all of its cache with distinct neighbours and will reach its defined maximum
soon after initialisation.
\subsubsection{Clustering Coefficient}
The local clustering coefficient of a node describes how many of its neighbours
are neighbours themselves. It  The local cluster coefficient $C_i$ of the node
$i$ is defined as follows:
\[C_i = \frac{2n}{k_i(k_i-1)}\]
where $k_i$ is the of number of neighbours of $i$ and $n$ is the number of edges
between the neighbors of $i$.
\subsubsection{Betweenness Centrality}
In order to describe how central a node is in a network one can use several
centrality metrics. The betweenness centrality of a node is calculated by
counting the number of shortest paths that run through that node. The higher the
number the more shortest paths rely on the connections of that node. This metric
can indicate the relative traffic load of a node. Assume that every node sends
the same amount of data over the network and that data takes the shortest
possible paths in the network. Since a node with a higher betweenness
centrality, is also part of more shortest paths, it will have to handle more
traffic than a node with a lower betweenness centrality. 

\subsubsection{Shortest Path Length}
The shortest path is the lowest number of hops between two different nodes
within a network. It indicates the time and cost of communication between those
two nodes. In a directed graph like the one Cyclon generates the shortest path
from one node to another is does not have to be the same as the shortest path in
the opposite direction. One way to accumulate the shortest path is to calculate
pick the maximum of all shortest paths from the local node. This number
indicates the number of hops needed to send a broadcast message to all
participants of the network. This could be again averaged or the median could be
picked over all nodes in the network to obtain a quality indicator of the whole
network. While the maximum of all shortest paths within the network is the
diameter of the network.
